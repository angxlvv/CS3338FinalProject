% Options for packages loaded elsewhere
\PassOptionsToPackage{unicode}{hyperref}
\PassOptionsToPackage{hyphens}{url}
\documentclass[
]{article}
\usepackage{xcolor}
\usepackage{amsmath,amssymb}
\setcounter{secnumdepth}{-\maxdimen} % remove section numbering
\usepackage{iftex}
\ifPDFTeX
  \usepackage[T1]{fontenc}
  \usepackage[utf8]{inputenc}
  \usepackage{textcomp} % provide euro and other symbols
\else % if luatex or xetex
  \usepackage{unicode-math} % this also loads fontspec
  \defaultfontfeatures{Scale=MatchLowercase}
  \defaultfontfeatures[\rmfamily]{Ligatures=TeX,Scale=1}
\fi
\usepackage{lmodern}
\ifPDFTeX\else
  % xetex/luatex font selection
\fi
% Use upquote if available, for straight quotes in verbatim environments
\IfFileExists{upquote.sty}{\usepackage{upquote}}{}
\IfFileExists{microtype.sty}{% use microtype if available
  \usepackage[]{microtype}
  \UseMicrotypeSet[protrusion]{basicmath} % disable protrusion for tt fonts
}{}
\makeatletter
\@ifundefined{KOMAClassName}{% if non-KOMA class
  \IfFileExists{parskip.sty}{%
    \usepackage{parskip}
  }{% else
    \setlength{\parindent}{0pt}
    \setlength{\parskip}{6pt plus 2pt minus 1pt}}
}{% if KOMA class
  \KOMAoptions{parskip=half}}
\makeatother
\usepackage{longtable,booktabs,array}
\usepackage{calc} % for calculating minipage widths
% Correct order of tables after \paragraph or \subparagraph
\usepackage{etoolbox}
\makeatletter
\patchcmd\longtable{\par}{\if@noskipsec\mbox{}\fi\par}{}{}
\makeatother
% Allow footnotes in longtable head/foot
\IfFileExists{footnotehyper.sty}{\usepackage{footnotehyper}}{\usepackage{footnote}}
\makesavenoteenv{longtable}
\usepackage{graphicx}
\makeatletter
\newsavebox\pandoc@box
\newcommand*\pandocbounded[1]{% scales image to fit in text height/width
  \sbox\pandoc@box{#1}%
  \Gscale@div\@tempa{\textheight}{\dimexpr\ht\pandoc@box+\dp\pandoc@box\relax}%
  \Gscale@div\@tempb{\linewidth}{\wd\pandoc@box}%
  \ifdim\@tempb\p@<\@tempa\p@\let\@tempa\@tempb\fi% select the smaller of both
  \ifdim\@tempa\p@<\p@\scalebox{\@tempa}{\usebox\pandoc@box}%
  \else\usebox{\pandoc@box}%
  \fi%
}
% Set default figure placement to htbp
\def\fps@figure{htbp}
\makeatother
\setlength{\emergencystretch}{3em} % prevent overfull lines
\providecommand{\tightlist}{%
  \setlength{\itemsep}{0pt}\setlength{\parskip}{0pt}}
\usepackage{bookmark}
\IfFileExists{xurl.sty}{\usepackage{xurl}}{} % add URL line breaks if available
\urlstyle{same}
\hypersetup{
  hidelinks,
  pdfcreator={LaTeX via pandoc}}

\author{}
\date{}

\begin{document}

Group 5

\begin{quote}
\textbf{Want2Remember}

Software Design Document

Name (s): Charlie Kaing, Danielle Garcia,Daniel Moncada Santana, Angel
Villagomez, Pablo Gonzalez Dominguez Lab Section:3338-02

Workstation:

Date: (05/02/2025)
\end{quote}

Table of
Contents..............................................................................................................\textless pg
\#2\textgreater{}

1.
Introduction..................................................................................................................\textless pg
\#3\textgreater{}

1.1.
Purpose...........................................................................................................\textless pg
\#3\textgreater{}

1.2.
Scope\ldots\ldots\ldots\ldots\ldots\ldots\ldots\ldots.............................................................................\textless pg
\#3\textgreater{}

1.3.
Overview.........................................................................................................\textless pg
\#3\textgreater{}

1.4. Reference
Material........................................................................................\textless pg
\#3\textgreater{}

1.5. Definitions and
Acronyms............................................................................\textless pg
\#3\textgreater{}

2.Design
Considerations...................................................................................................\textless pg
\#4\textgreater{}

3. System
Architecture.....................................................................................................\textless pg
\#5\textgreater{}

3.1. Architectural
Design......................................................................................\textless pg
\#5\textgreater{}

3.2. Decomposition
Description...........................................................................\textless pg
\#5\textgreater{}

3.3. Design
Rationale............................................................................................\textless pg
\#5\textgreater{}

4. Data
Design...................................................................................................................\textless pg
\#5\textgreater{}

4.1.Data
Description.............................................................................................\textless pg
\#5\textgreater{}

4.2.Data
Dictionary..............................................................................................\textless pg
\#5\textgreater{}

5. Componet
Design.......................................................................................................\textless pg
\#6\textgreater{}

6. Detailed System
Design..............................................................................................\textless pg
\#6\textgreater{}

6.1.
Responsibilities..............................................................................................\textless pg
\#6\textgreater{}

6.2.
Constrants......................................................................................................\textless pg
\#6\textgreater{}

6.3.Compoition....................................................................................................\textless pg
\#6\textgreater{}

7. Requirements
Matrix....................................................................................................\textless pg
\#7\textgreater{}

8.
Appendices....................................................................................................................\textless pg
\#7\textgreater{}

\textbf{Revision History}

\begin{longtable}[]{@{}
  >{\raggedright\arraybackslash}p{(\linewidth - 6\tabcolsep) * \real{0.1793}}
  >{\raggedright\arraybackslash}p{(\linewidth - 6\tabcolsep) * \real{0.1810}}
  >{\raggedright\arraybackslash}p{(\linewidth - 6\tabcolsep) * \real{0.5220}}
  >{\raggedright\arraybackslash}p{(\linewidth - 6\tabcolsep) * \real{0.1178}}@{}}
\toprule\noalign{}
\begin{minipage}[b]{\linewidth}\centering
Name
\end{minipage} & \begin{minipage}[b]{\linewidth}\centering
Date
\end{minipage} & \begin{minipage}[b]{\linewidth}\centering
Reason For Changes
\end{minipage} & \begin{minipage}[b]{\linewidth}\centering
Version
\end{minipage} \\
\begin{minipage}[b]{\linewidth}\centering
Charlie
\end{minipage} & \begin{minipage}[b]{\linewidth}\centering
3-23
\end{minipage} & \begin{minipage}[b]{\linewidth}\centering
Added format and started adding information to all sections
\end{minipage} & \begin{minipage}[b]{\linewidth}\centering
1.1
\end{minipage} \\
\begin{minipage}[b]{\linewidth}\centering
Danielle
\end{minipage} & \begin{minipage}[b]{\linewidth}\centering
4-5
\end{minipage} & \begin{minipage}[b]{\linewidth}\centering
Added to sections 1- 5
\end{minipage} & \begin{minipage}[b]{\linewidth}\centering
1.2
\end{minipage} \\
\begin{minipage}[b]{\linewidth}\centering
Daniel
\end{minipage} & \begin{minipage}[b]{\linewidth}\centering
5-5
\end{minipage} & \begin{minipage}[b]{\linewidth}\centering
Added to sections 7-8
\end{minipage} & \begin{minipage}[b]{\linewidth}\centering
1.3
\end{minipage} \\
\begin{minipage}[b]{\linewidth}\centering
\end{minipage} & \begin{minipage}[b]{\linewidth}\centering
\end{minipage} & \begin{minipage}[b]{\linewidth}\centering
\end{minipage} & \begin{minipage}[b]{\linewidth}\raggedright
\end{minipage} \\
\midrule\noalign{}
\endhead
\bottomrule\noalign{}
\endlastfoot
\end{longtable}

\textbf{1. INTRODUCTION}

\begin{quote}
\textbf{1.1 Purpose}

This Software Design Document is to provide the documentation for the
Want2Remember app for CS3338. This includes scope, references and API
used for the development of this project. The purpose is to breakdown,
analyze and document the workflow used in the creation of the app
Want2Remeber.

\textbf{1.2 Scope}

This document provides an overview of the design and development of
Want2Remember including available resources and procedures for the
developers. It will focus on the development of Want2Remember, not so
much its functionality. This project aims to explore the tools used for
development such as Github and GitHub.

\textbf{1.3 Overview}
\end{quote}

The Want2Remember project is a web and mobile application made to assist
users with any memory related challenges by helping them store memories,
contacts, reminders, or personal data. Want2Remeber uses Components
based architecture to allow for modularity in its reuse and complexity.
This document attempts to model how the development team may have
organized their work and responsibilities during its creation.

\begin{quote}
\textbf{1.4 Reference Material}

List any documents, if any, which were used as sources of information
for the test plan.
\end{quote}

\begin{itemize}
\item
  Want2Remember Software Design Document
\item
  Want2Remember Software Requirements Specification
\item
  Want2Remeber Testrail report
\end{itemize}

\begin{quote}
\textbf{1.5 Definitions and Acronyms}

Provide definitions of all terms, acronyms, and abbreviations that might
exist to properly interpret the SDD. These definitions should be items
used in the SDD that are most likely not known to the audience.
\end{quote}

\begin{itemize}
\item
  SDD: Software Development Document
\item
  SRS: Software Requirements Specifications
\item
  UI: User Interface
\end{itemize}

\begin{quote}
\textbf{2. Design Considerations}

The team behind Want2Remember used more modern software development
practices that are used in academic and industry-sponsored settings. The
application was built over the span of several months and involved using
agile principles with multiple coders/engineers working on separate
components

\textbf{2.1 Assumptions and dependencies}

The current software that we are planning to use is:
\end{quote}

\begin{itemize}
\item
  Python
\item
  Tkinter
\item
  Github
\item
  TestRail
\end{itemize}

Software original creators used:

\begin{itemize}
\item
  GitHub
\item
  FireBase
\item
  Visual Studio Code
\item
  React/React Native
\item
  Jira
\end{itemize}

\begin{quote}
\textbf{2.2 General Constraints}
\end{quote}

\begin{itemize}
\item
  No direct access to code or team sprint progress
\item
  Limited information on team communication apart from where conducted
\item
  As a project based on replication we cannot fully replicate the team
  structure or any changes over time
\end{itemize}

\begin{quote}
\textbf{2.3 Goals and Guidelines}

The goal of this project is to learn about the uses of software
engineering tools such as github in a team. We hope to identify the
relationships between the components and responsibilities while also
learning how a professional development is managed in a more
collaborative environment.

\textbf{3. SYSTEM ARCHITECTURE}

\textbf{3.1 Architectural Design}
\end{quote}

\begin{itemize}
\item
  Reminders Component: The reminders are stored as a data type that can
  interfaced with other components
\item
  GUI Component: Contains libraries and software to create the GUI
  interface.
\item
  Time Component: Keeps track of calendar reminders may be a sub
  component of Reminders
\item
  Contacts
\end{itemize}

\begin{quote}
\textbf{4. System Architecture}

\textbf{4.1 Data Description}

The current plans for storage are to at first use a standard list to
store reminders and to retrieve them. We are also investigating our data
structure that may improve our project performance such as trees and
database integration such as mongoDB or mysql to have persistence.

\textbf{4.2 Architecture Diagram}
\end{quote}

\includegraphics[width=4.73022in,height=2.65978in]{media/image1.png}

\begin{quote}
\textbf{5. Policies and Tactics}

\textbf{5.1 Tools used}
\end{quote}

\begin{itemize}
\item
  Version Control:
\item
  Github
\end{itemize}

\begin{itemize}
\item
  Jira
\item
  Editor:
\item
  Visual Studio Code
\end{itemize}

\begin{quote}
\textbf{6. Detailed System Design}

\textbf{6.1 Responsibilities}

GUI Component: The GUI component is responsible for all Graphic
interfaces code such libraries and interfaces to other components and
allows extensions of additional functionality to future components.
\end{quote}

\begin{itemize}
\item
  \textbf{Reminder Components:} Manages reminder times, tags, and any
  associated memory data
\item
  \textbf{GUI Component}: Displays the main interface, manages any
  routes/navigations, and renders the notifications
\item
  \textbf{Data Component:} Handles the reading/writing to firebase
\end{itemize}

\begin{quote}
\textbf{6.2 Constraints}

Want2Remember currently does not have any persistence or storage when
the program closes all reminders and schedules are deleted.

\textbf{6.3 Composition}
\end{quote}

\begin{itemize}
\item
  Github is currently being used for repository storage and to help with
  collaboration of multiple members of the team.
\item
  Communication was done/occured on Discord
\item
  Jira was used to assign, track, and manage any team assigned tasks
\end{itemize}

\begin{quote}
\textbf{7. REQUIREMENTS MATRIX}

Provide a cross reference that traces components and data structures to
the requirements in your SRS document. Use a tabular format to show
which system components satisfy each of the functional requirements from
the SRS. Refer to the functional requirements by the numbers/codes that
you gave them in the SRS.
\end{quote}

\begin{longtable}[]{@{}
  >{\raggedright\arraybackslash}p{(\linewidth - 6\tabcolsep) * \real{0.1909}}
  >{\raggedright\arraybackslash}p{(\linewidth - 6\tabcolsep) * \real{0.2466}}
  >{\raggedright\arraybackslash}p{(\linewidth - 6\tabcolsep) * \real{0.3024}}
  >{\raggedright\arraybackslash}p{(\linewidth - 6\tabcolsep) * \real{0.2601}}@{}}
\toprule\noalign{}
\begin{minipage}[b]{\linewidth}\raggedright
Requirement ID
\end{minipage} & \begin{minipage}[b]{\linewidth}\raggedright
Requirement Description
\end{minipage} & \begin{minipage}[b]{\linewidth}\raggedright
SDD Component(s)
\end{minipage} & \begin{minipage}[b]{\linewidth}\raggedright
Data Structure(s)
\end{minipage} \\
\begin{minipage}[b]{\linewidth}\raggedright
1
\end{minipage} & \begin{minipage}[b]{\linewidth}\raggedright
Home Screen
\end{minipage} & \begin{minipage}[b]{\linewidth}\raggedright
Reminder Component, GUI Component
\end{minipage} & \begin{minipage}[b]{\linewidth}\raggedright
Python List of reminder objects
\end{minipage} \\
\begin{minipage}[b]{\linewidth}\raggedright
2
\end{minipage} & \begin{minipage}[b]{\linewidth}\raggedright
Create screen
\end{minipage} & \begin{minipage}[b]{\linewidth}\raggedright
Reminder Component, GUI Component Time Component
\end{minipage} & \begin{minipage}[b]{\linewidth}\raggedright
Python Dictionary and Python List
\end{minipage} \\
\begin{minipage}[b]{\linewidth}\raggedright
3
\end{minipage} & \begin{minipage}[b]{\linewidth}\raggedright
Edit Screen
\end{minipage} & \begin{minipage}[b]{\linewidth}\raggedright
Reminder Component, Time Component
\end{minipage} & \begin{minipage}[b]{\linewidth}\raggedright
Python List of the different Dictionaries
\end{minipage} \\
\begin{minipage}[b]{\linewidth}\raggedright
4
\end{minipage} & \begin{minipage}[b]{\linewidth}\raggedright
More Details Screen
\end{minipage} & \begin{minipage}[b]{\linewidth}\raggedright
Reminder Component, GUI
\end{minipage} & \begin{minipage}[b]{\linewidth}\raggedright
Python List of reminder objects
\end{minipage} \\
\begin{minipage}[b]{\linewidth}\raggedright
5
\end{minipage} & \begin{minipage}[b]{\linewidth}\raggedright
Contacts Screen
\end{minipage} & \begin{minipage}[b]{\linewidth}\raggedright
Contacts Component
\end{minipage} & \begin{minipage}[b]{\linewidth}\raggedright
Python Dictionary and Python List
\end{minipage} \\
\begin{minipage}[b]{\linewidth}\raggedright
6
\end{minipage} & \begin{minipage}[b]{\linewidth}\raggedright
Search Screen
\end{minipage} & \begin{minipage}[b]{\linewidth}\centering
Reminder Component, GUI Component, Cloud Firebase
\end{minipage} & \begin{minipage}[b]{\linewidth}\raggedright
Firebase Query
\end{minipage} \\
\begin{minipage}[b]{\linewidth}\raggedright
7
\end{minipage} & \begin{minipage}[b]{\linewidth}\raggedright
Settings Screen
\end{minipage} & \begin{minipage}[b]{\linewidth}\raggedright
GUI Component
\end{minipage} & \begin{minipage}[b]{\linewidth}\raggedright
JSON or Config files
\end{minipage} \\
\begin{minipage}[b]{\linewidth}\raggedright
8
\end{minipage} & \begin{minipage}[b]{\linewidth}\raggedright
Backend Service
\end{minipage} & \begin{minipage}[b]{\linewidth}\raggedright
Cloud (Firebase)
\end{minipage} & \begin{minipage}[b]{\linewidth}\raggedright
Firebase Cloud
\end{minipage} \\
\midrule\noalign{}
\endhead
\bottomrule\noalign{}
\endlastfoot
\end{longtable}

\begin{quote}
\textbf{8. APPENDICES}

Appendices may be included, either directly or by reference, to provide
supporting details that could aid in the understanding of the Software
Design Document. F
\end{quote}

\end{document}
