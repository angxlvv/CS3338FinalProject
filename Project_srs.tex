% Options for packages loaded elsewhere
\PassOptionsToPackage{unicode}{hyperref}
\PassOptionsToPackage{hyphens}{url}
\documentclass[
]{article}
\usepackage{xcolor}
\usepackage{amsmath,amssymb}
\setcounter{secnumdepth}{-\maxdimen} % remove section numbering
\usepackage{iftex}
\ifPDFTeX
  \usepackage[T1]{fontenc}
  \usepackage[utf8]{inputenc}
  \usepackage{textcomp} % provide euro and other symbols
\else % if luatex or xetex
  \usepackage{unicode-math} % this also loads fontspec
  \defaultfontfeatures{Scale=MatchLowercase}
  \defaultfontfeatures[\rmfamily]{Ligatures=TeX,Scale=1}
\fi
\usepackage{lmodern}
\ifPDFTeX\else
  % xetex/luatex font selection
\fi
% Use upquote if available, for straight quotes in verbatim environments
\IfFileExists{upquote.sty}{\usepackage{upquote}}{}
\IfFileExists{microtype.sty}{% use microtype if available
  \usepackage[]{microtype}
  \UseMicrotypeSet[protrusion]{basicmath} % disable protrusion for tt fonts
}{}
\makeatletter
\@ifundefined{KOMAClassName}{% if non-KOMA class
  \IfFileExists{parskip.sty}{%
    \usepackage{parskip}
  }{% else
    \setlength{\parindent}{0pt}
    \setlength{\parskip}{6pt plus 2pt minus 1pt}}
}{% if KOMA class
  \KOMAoptions{parskip=half}}
\makeatother
\usepackage{longtable,booktabs,array}
\usepackage{calc} % for calculating minipage widths
% Correct order of tables after \paragraph or \subparagraph
\usepackage{etoolbox}
\makeatletter
\patchcmd\longtable{\par}{\if@noskipsec\mbox{}\fi\par}{}{}
\makeatother
% Allow footnotes in longtable head/foot
\IfFileExists{footnotehyper.sty}{\usepackage{footnotehyper}}{\usepackage{footnote}}
\makesavenoteenv{longtable}
\ifLuaTeX
  \usepackage{luacolor}
  \usepackage[soul]{lua-ul}
\else
  \usepackage{soul}
\fi
\setlength{\emergencystretch}{3em} % prevent overfull lines
\providecommand{\tightlist}{%
  \setlength{\itemsep}{0pt}\setlength{\parskip}{0pt}}
\usepackage{bookmark}
\IfFileExists{xurl.sty}{\usepackage{xurl}}{} % add URL line breaks if available
\urlstyle{same}
\hypersetup{
  hidelinks,
  pdfcreator={LaTeX via pandoc}}

\author{}
\date{}

\begin{document}

\textbf{Software Requirements Specification}

\textbf{Revision History}

\begin{longtable}[]{@{}
  >{\raggedright\arraybackslash}p{(\linewidth - 6\tabcolsep) * \real{0.1443}}
  >{\raggedright\arraybackslash}p{(\linewidth - 6\tabcolsep) * \real{0.2440}}
  >{\raggedright\arraybackslash}p{(\linewidth - 6\tabcolsep) * \real{0.2743}}
  >{\raggedright\arraybackslash}p{(\linewidth - 6\tabcolsep) * \real{0.3374}}@{}}
\toprule\noalign{}
\begin{minipage}[b]{\linewidth}\raggedright
\textbf{Version}
\end{minipage} & \begin{minipage}[b]{\linewidth}\raggedright
\textbf{Date}
\end{minipage} & \begin{minipage}[b]{\linewidth}\raggedright
\textbf{Author}
\end{minipage} & \begin{minipage}[b]{\linewidth}\raggedright
\textbf{Change Description}
\end{minipage} \\
\begin{minipage}[b]{\linewidth}\raggedright
1.0
\end{minipage} & \begin{minipage}[b]{\linewidth}\raggedright
\end{minipage} & \begin{minipage}[b]{\linewidth}\raggedright
\end{minipage} & \begin{minipage}[b]{\linewidth}\raggedright
\end{minipage} \\
\begin{minipage}[b]{\linewidth}\raggedright
\end{minipage} & \begin{minipage}[b]{\linewidth}\raggedright
\end{minipage} & \begin{minipage}[b]{\linewidth}\raggedright
\end{minipage} & \begin{minipage}[b]{\linewidth}\raggedright
\end{minipage} \\
\midrule\noalign{}
\endhead
\bottomrule\noalign{}
\endlastfoot
\end{longtable}

\textbf{TABLE OF CONTENTS}

\begin{quote}
\hyperref[section-1.-introduction]{\ul{INTRODUCTION}}

\hyperref[_ixwp5se860rt]{\ul{Purpose}}

\hyperref[documentation-conventions]{\ul{Scope}}

\hyperref[intended-audience]{\ul{Overview}}

\hyperref[_khs32z7vtz5h]{\ul{Definitions, Acronyms, and Abbreviations}}

\hyperref[_frw4je31to8t]{\ul{References}}

\hyperref[section-2.-general-description]{\ul{General Description}}

\hyperref[system-analysis]{\ul{Product Perspective}}

\hyperref[product-perspective]{\ul{Product Functions}}

\hyperref[product-function]{\ul{User Characteristics}}

\hyperref[general-constraints]{\ul{General Constraints}}

\hyperref[assumptions-and-dependencies]{\ul{Assumptions and
Dependencies}}

\hyperref[section-3.-requirements-specification]{\ul{Requirements}}

\hyperref[functional-requirements]{\ul{Functional Requirements}}

\hyperref[external-interface-requirements]{\ul{External Interface
Requirements}}

\hyperref[_jb5ksphe16w2]{\ul{User Interfaces}}

\hyperref[_eo0p461hhe3r]{\ul{Hardware Interfaces}}

\hyperref[_fpu3ytfo7r1]{\ul{Software Interfaces}}

\hyperref[_dxgri1jo8rz6]{\ul{Communications Interfaces}}

\hyperref[_ucoyn2yvl6c3]{\ul{Performance Requirements}}

\hyperref[design-constraints]{\ul{Design Constraints}}

\hyperref[_kj9rygdikkwy]{\ul{Standards Compliance}}

\hyperref[_i0u468c6zuoi]{\ul{Hardware Limitations}}

\hyperref[nonfunctional-requirements]{\ul{Attributes}}

\hyperref[performance]{\ul{Availability}}

\hyperref[security]{\ul{Security}}

\hyperref[_e5po9qanxzue]{\ul{Maintainability}}

\hyperref[_rj4tlkisvdbj]{\ul{Transferability/Conversion}}

\hyperref[other-requirements]{\ul{Other Requirements}}

\hyperref[legal-requirements]{\ul{Operations}}

\hyperref[ethical-requirements]{\ul{Site Adaptation}}

\hyperref[appendixes]{\ul{Appendixes}}

\hyperref[_clp35ii7xzat]{\ul{Annex 1: Characteristics of a good SRS}}

\hyperref[_l587f99ytp53]{\ul{Annex 2: Organizing the specific
requirements}}

\hyperref[_lg3ompi10x86]{\ul{Annex 3: SRS templates}}

\hyperref[_fq84tbgccxb3]{\ul{Template of SRS Section 3 organized by
feature}}

\hyperref[_vpigomc8pdzb]{\ul{Template of SRS Section 3 organized by user
class}}

\hyperref[_8avjwjy42bpe]{\ul{Template of SRS Section 3 showing multiple
organizations}}
\end{quote}

\section{Section 1. INTRODUCTION}\label{section-1.-introduction}

\section{Purpose}\label{purpose}

This document is for identification of requirements for Want2Remember
app by giving detailed explanation of the application and its purpose
and how will users interact with the software

\subsection{Documentation Conventions}\label{documentation-conventions}

This document is intended for developers and the project managers as
well as users to help understand requirements and parse information for
the software.

\subsection{Intended Audience}\label{intended-audience}

Want2Remeber application is intended for users have some sort of mental
or cognitive impairment that prevents from performing daily task

\section{Section 2. General
Description}\label{section-2.-general-description}

\subsection{System Analysis}\label{system-analysis}

People with Cognitive impairments would need assistance when it comes to
task such as appointments,routines and medical treatment. The software
should be able notify daily task, create schedules and sensitive data

\subsection{Product Perspective}\label{product-perspective}

Want2Remeber has a basic backend command line interface that is used for
testing reminders including scheduling. A GUI interface that has
features commonly seen in calendar and reminders apps is planned.

\subsection{Product Function}\label{product-function}

Home Menu: Initial Landing screen for access to current reminders as
well as create or delete new one

Reminders menu: Shows a list of reminders including dates and time as
well as editing existing reminders

Create/delete Menu: allows users to create reminders and delete them

\subsection{General Constraints}\label{general-constraints}

Python is the current language being used for the implementation of
want2Remember any device that can run a python interpreter can run the
Want2remeber

\subsection{Assumptions and
Dependencies}\label{assumptions-and-dependencies}

It is generally assume that users may be able to access and use a device
that can run a python interpreter

\section{Section 3. Requirements
Specification}\label{section-3.-requirements-specification}

\subsection{Functional Requirements}\label{functional-requirements}

Home Screen: The interface should be able to show all actions that a
users could take.

Reminder Screen: The Reminders screen shall be able to show at a glance
all reminders that a User has made

Create Screen: The Create menu should allow for creation and deletion of
reminders

Edit Screen: The Edit menu shall allow the user to editor delete a
memory

More Details Screen:The More Details menu shall allow the user to view
the details of a screen

Contacts Screen:The Contacts menu shall display contact quick looks and
allow for editing contacts

Search Screen:The Search menu shall allow the user to search through the
database with the search bar.

Settings Screen:The Settings menu shall allow the user the ability to
import/export JSON format memories.

Backend Screen:The Beckend shall allow synchronization between caregiver
and care receiver data

\subsection{External Interface
Requirements}\label{external-interface-requirements}

Not Applicable

\subsection{Design Constraints}\label{design-constraints}

Interfaces need to be able to display information clearly.

Various different devices may need to be adapted into different factors
for mobile or desktop

\subsection{Nonfunctional
Requirements}\label{nonfunctional-requirements}

\subsubsection{Performance}\label{performance}

Notifications need to be done on time when they schedule. Delays should
be more than 1 minute and ideally should be instant.

\subsubsection{Security}\label{security}

Currently as a offline software Want2remember shouldn't need to many
security requirements however, if online components are to be made there
should be more security requirements should be planned

\subsection{Other Requirements}\label{other-requirements}

\subsubsection{Legal Requirements}\label{legal-requirements}

Want2remember should be secure and protective of user data.

Only users should be able to access personal data of their reminders.

Personal Data should only be shared with users consent

\subsubsection{Ethical Requirements}\label{ethical-requirements}

Want2remeber purpose is to help cognitive impaired individuals every
aspect of the program to accommodate their needs.

User data must be secure only shared with consent of the user

\section{Appendixes}\label{appendixes}
GUI: Graphical User Interface
JSON: JavaScript Object Notation

\end{document}
